\subsection{GGT Chapter 7.1 - Finitely Generated Groups}
    \begin{definition}[Finitely Generated]
        We say a group, $G$, is \textit{finitely generated} if it has a finite \textit{generating set},
    \end{definition} 
    Moreover we use rank($G$) to denote the minimal number of generators for $G$.
    \begin{example}
        We have that $(\ZZ,+)$ is finitely generated by {1} and {-1}. Moreover we have that it can be generated by coprime integers {$p,q$}
    \end{example}
    \begin{example}
        The Group $(\QQ,+)$ is not finitely generated
    \end{example}
    \BBreak
    \subsubsection{Wreath Product}
    \begin{definition}[Group Complement]
        Let $K$ be a subgroup of $G$, then we say that $Q$ is a the \textit{complement} to $K$ is $Q\cap K = \{1_G\}$ and $QK = G$
        
    \end{definition}
    \begin{definition}[Semi-Direct Product]
        Define the \textit{Semi-Direct Product} for two groups $(G,\cdot_G)$ and $(H,\cdot_H)$, and some homomorphism $\varphi: H\to G$ as $G\rtimes_\varphi H$. Which is the group who's underlying set is $G\times H = \{(g,h)\vert g\in G, h\in H\}$, and who's operation is defined to be 
        \[(g_1,h_1)\star(g_2,h_2) = (g_1(\varphi(h_1)g_2),h_1h_2)\]
    \end{definition}
    \begin{exercise}
        Prove that this is a group.
    \end{exercise}
    \begin{definition}[Wreath Product]
    YO THIS THING IS SO CONFUSING
    \end{definition}
    \begin{example}[Lamplighter Group]
        We define the \href{https://en.wikipedia.org/wiki/Lamplighter_group}{Lamp Lighter Group} to be the \textit{restricted} Wreath Product between $\ZZ^2\wr\ZZ$
    \end{example}
    \BBreak
    \begin{proposition}
        Every Quotient, $\overline{G}$, of a finitely generated group, $G$, is finitely generated
    \end{proposition}
    We may take the generators of $\overline{G}$ to be the images of the generators of $G$
    \begin{proposition}
        If $N$ is a normal subgroup of $G$, with $N$ and $G\backslash N$ being finitely generated, then we have that $G$ is also finitely generated
    \end{proposition}
    By assumption we have that $\{n_1,\ldots,n_k\}$ and $\{g_1N,\ldots,g_m\}$ generate $N$ and $G\backslash N$ respectively. Then we have that
    \begin{center}
        $\{g_i,n_j\vert1\leq i\leq m$, $1\leq j\leq k\}$
    \end{center}
    Generates $G$
    \BBreak
    \subsubsection{Special Homomorphism}
    \begin{definition}[Monomorphism]
        A \textit{monomorphism} is an injective homomorphism
    \end{definition}
    \begin{definition}[Epimorphism]
        An \textit{epimorphism} is a surjective homomorphism
    \end{definition}
    \begin{definition}[Automorphism]
        An \textit{automorphism} is a bijective homomorphism from a Group (more generally an algebraic structure) to itself
    \end{definition}
    \BBreak
    \subsubsection{Exact Sequences}
    \begin{definition}[Exact Sequences]
        A sequence of groups and group homomorphisms is said to be \textit{exact} if the the image of each homomorphism is equal to the kernel of the next
    \end{definition}
    \begin{remark}
        There also exists \href{https://en.wikipedia.org/wiki/Exact_sequence#Long_exact_sequence}{Long Exact Sequences}
    \end{remark}
    We say an exact sequence is \textit{short} if it is of form
    \[1\to A\stackrel{f}{\to}B\stackrel{g}{\to} C\to 1\]
    Furthermore we have that $f$ is a monomorohism and $g$ is an epimorphism
    \BBreak
    \begin{lemma}
        If we have a short exact sequence of groups
        \[1\to A\stackrel{f}{\to}B\stackrel{g}{\to} C\to 1\]
        such that $A$ and $C$ are finitely generated, then $B$ is also finitely generated
    \end{lemma}
    \BBreak
    \subsubsection{Frattini Subgroups}
    \begin{definition}[Non-Generator]
        We say an element $x$ is a \textit{non-generator} of a group $G$, if for all generator sets $S$, the set $S\backslash\{x\}$ still generates $G$
    \end{definition}
    Then the set of all non-generators forms a subgroup of $G$ called the \textit{Frattini Subgroup}
    \BBreak
    \begin{definition}[Bounded Property]
        A group $G$ is said to have the \textit{bounded property} (or to be \textit{boundedly generated}) if there exists a finite subset $S=\{t_1,\ldots,t_m\}\subset G$ such that for all elements $g\in G$, $g$ can be written as
        \[g=t_1^{k_1}\ldots t_m^{k_m}\]
        where $k_1,k_2,\ldots,k_m$ are integers
        
    \end{definition}
    We have that all finitely generated abelian groups have this property, we also have that all finite groups have this property too. But we have that non-abelian \textit{free} groups do not have this property.
\subsection{GGT Chapter 7.2 - Free Groups}
Let $X$ be a set, then we call the elements in $X$ \textit{letters}, then defining the set
\[X^{-1}=\{a^{-1}\vert a\in X\}\]
to be the set of \textit{inverse letters}. We can think of $X\cup X^{-1}$ as an \textit{alphabet} then construct \textit{words} from it
\begin{definition}[Words]
    A \textit{word} in $X\cup X^{-1}$ is a finite string of letters from $X\cup X^{-1}$.
\end{definition}
We notice that a word could be an empty string, in such a case we let $1$ denote the empty word. Further more we define the \textit{length} of a word to be the amount of letters in that word. Naturally, the empty word has length 0. Furthermore we use $X^*$ to denote the set of words of the alphabet $X\cup X^{-1}$, where 1 is used to denote the empty word.
\begin{definition}[Reduced]
    We say a word in $X^*$ is \textit{reduced} if there are no pairs of consecutive letters of form $a^{-1}a$ or $aa^{-1}$. Then the \textit{reduction} of a word is the deletion of all the consecutive letters of form $a^{-1}a$ or $aa^{-1}$.
\end{definition}
\begin{example}
    Given that $a$ and $b$ aren't inverses of each other. The word $abcc^{-1}ab$ is not reduced, and $abab$ is the reduction of the word
\end{example}
\begin{example}
    The word $abc$ is reduced
\end{example}

More generally we say that word is \textit{cyclically reduced} if it is reduced and the first and last letters aren't inverses of each other.
\begin{example}
    The word $abca^{-1}$ is reduced by not cyclically reduced
\end{example}
\begin{definition}[Conjugation]
    We may conjugate a word $w\in X^*$ by some letter $a\in X\cup X^{-1}$ by
    \[ w'=a^{-1}wa\]
    Which results in a word $w'$ with reduction of length greater than $w$
\end{definition}
Furthermore we may define an equivalent relation on $X^*$ such that we say two elements, $w\sim w'$ are equivalent if $w$ can be obtained from $w'$ by a finite sequence of reductions and their ``inverse reduction"
\begin{proposition}
    Every word $w\in X^*$ is equivalent to some \textit{unique} reduced word in $X^*$
\end{proposition}
Moreover we could identify $X^*/\sim$ with $F(X)$
\begin{definition}[Free Group]
    The \textit{free group} over $X$ is the set $F(X)$ with operation $*$ defined by $w*w'$ is the unique reduced word equivalent to $ww'$
\end{definition}
The cardinality of $X$ is called the \textit{rank} of $F(X)$. Although now we have two definitions for \textit{rank}, the cardinality of a set, and the size of a minimal generator set for a finitely generated group. It actually turns out that both of these numbers are the same.
\begin{exercise}
    A free group of rank at least 2 is not abelian. Thus, \textit{free non-abelian} means ‘free of rank at least 2.’
\end{exercise}
let $G$ be a free group of rank at least 2, we notice that the operation on $G$ is concatenation, so by letting $a,b\in G$ be arbitrary such that $a$ and $b$ aren't inverses of each other. Then it is so fucking obvious that $ab$ isn't equal to $ba$. So $G$ isn't abelian $\qed$
\BBreak
\subsubsection{Semigroup}
\begin{definition}[Semigroup]
    A set, $S$, with a binary operation $*$ that satisfies associativity
    \begin{itemize}
        \item $(x*y) * z = x * (y*z)$ for all $x,y,z\in S$
    \end{itemize}
    is called a \textit{semigroup}
\end{definition}
\BBreak
The \textit{free semigroup}, $F^s(X)$, over $X$ is defined synonymous to the definition of $F(X)$. But instead we take letters from only $X$, and not $X^{-1}$. With this construction we don't need to worry about reductions
\begin{proposition}
    \hypertarget{Prop1_27}{}A map $\varphi:X\to G$ from a set $X$ to a group $G$ can be extended to a homomorphism $\Phi :F(X)\to G$, furthermore this extension is unique
\end{proposition}
We first extend $\varphi$ to $X\cup X^{-1}$ by defining $\varphi(a^{-1})=\varphi(a)^{-1}$. Then for every reduced word in $w\in F(x)$, define 
\[ \Phi(w) = \varphi(a_1)\ldots\varphi(a_n)\]
Finally we define $\Phi(1_F) = 1_G$. Then we have our homomorphism. Then we prove Uniqueness. Let $\Psi(x):F(X)\to G$ be a homomorphism such that $\Psi(x) =\varphi(x)$, then we have
\[\Psi(x) = \varphi(a_1)\ldots\varphi(a_n) = \Phi(x)\]
\noindent So we have a homomorphism that is unique$\qed$
\begin{corollary}
    Every group is a \href{https://en.wikipedia.org/wiki/Quotient_group#Definition}{\textit{quotient group}} of a free group
\end{corollary}
The idea is to apply the previous proposition but take $X$ to be a generating set for $G$
\begin{definition}[Split]
    We say the short exact sequence
    \[1\to A\to B\to C\to 1\]
    \textit{splits} if $B= A\bigoplus C$
\end{definition}
\begin{lemma}
    Every short exact sequence \textit{splits}
    \[1\to N\to G\stackrel{f}{\to} F(X)\to 1\]
    Furthermore we have that there exists a subgroup of $G$ ismoorphic to $F(X)$
\end{lemma}
\begin{corollary}
    Every short exact sequence 
    \[1\to G\to N\to\ZZ\to1\]
    \textit{splits}
\end{corollary}
\subsection{GGT Chapter 7.3 - Presentation of Groups}
Let $G$ be a group with generating set $S$, then by \hyperlink{Prop1_27}{Proposition 1.27} we may extend the inclusion map $i:S\to G$ to a unique epimorphism $\pi_S:F(S)\to G$, then we define the elements of $Ker(\pi_S)$ to \textit{relators} (or \textit{relations}) on the group $G$ with generating set $S$. The idea behind this is that we basically define some word to be equal to the identity. Moreover if we define $R=\{r_i\vert i\in I\}\subset F(S)$ such that $Ker(\pi_s)$ is \textit{normally generated} by $R$, then we have that the tuple $(S,R)$, usually denoted $\langle S\vert R\rangle$, is a presentation of $G$. We further define the elements $r\in R$ to be \textit{defining relators} of the presentation $\langle S\vert R\rangle$

\begin{definition}[Finitely Presented]
    if $S,R$ are both finite, then we say that the pair $(S,R)$ \textit{finitely represent} $G$. Moreover, we say a group, $G$, is \textit{finitely presented} if it's presentation is finite, which means that there are finite generators and finite relations
\end{definition}
\begin{definition}[Normally Generated]
    A subset $X\subseteq G$ normally generated $G$ if every $g\in G$ can be expressed as
    \begin{center}
        $g=a^{-1}xa$ for some $a\in G$ and $x\in X$
    \end{center}
\end{definition}
\begin{example}
    The cyclic group of order $n$ can be \textit{finitely} represented by
    \[\langle a \vert a^n=1\rangle\]
\end{example}
We have that our generating set is only one element, $a$, and our relation set is also one element $a^n$. So this group is \textit{finitely represented}.
\begin{definition}[Commutator]
    For two elements $g,h$ in a group $G$, we define their \textit{commutator} by
    \[[g,h] = g^{-1}h^{-1}gh\]
    The significance of this is that in $G$, we have that $g$ and $h$ commute
\end{definition}
You might've also noticed that representation's of groups aren't always unique, taking $n=6$ in the above example yields \href{https://en.wikipedia.org/wiki/Cyclic_group}{$C_6$}. But the following example is also $C_6$
\begin{example}
    The Cyclic group of order 6 can be represented as
    \[\langle x,y\vert x^2,y^3,[x,y]\rangle\]
    Where $[x,y]$ is used to denote the commutator
\end{example}
Although this example seems super abstract and out of the blue, there is some logic to it. We first notice that since we have $[x,y]$ in our relation set, we have that $x,y$ commute within our group. Then we may rewrite any element in $G$ to
\[x^ny^m\]
where $n\in\{0,1\}$ and $m\in\{0,1,2\}$. Then you may interpret $x$ as a generator of $C_2$ and $y$ as a generator of $C_3$. Then after taking their direct product it is trivial to see that it is isomorphic to $C_6$. Not the most rigorous explanation but this is how I understood it.
\begin{remark}
    It is often convention to drop the $``=1"$ from the relation, so the above example could be also written as
     \[\langle a \vert a^n\rangle\]
\end{remark}
\begin{exercise}
    Let $p$ be a prime number, let $\langle S,R\rangle$ be a finite presentation for $C_p$. Prove or Disprove\footnote{I have no idea if this is true or not, just a fun question I thought of} : There exists proper subgroups $G<C_p$ and $H<C_p$ such that $GH = C_p$
\end{exercise}
\begin{example}
    Let $A=\{a_1,a_2,\ldots\}$ be a set, then the free group $F(A)$ which is subject to $a_i^2=1$ is an \textit{non-finitely presented} group
\end{example}
This should be obvious as both our generating set and relations set have infinite elements.
\begin{remark}
    Sometimes finding a \textit{finite presentation} of a finitely presented group is difficult, and sometimes algorithmically impossible
\end{remark}
Furthermore we can create a group with representation by quotienting the free group of an alphabet by a set of relations
\[ G:=F(S)/Ker(\pi_S)\]
Then we have that $\langle S,R\rangle$ is a presentation of $G$. As a slight abuse of notation we simply write
\[ G=\langle S,R\rangle \]
\begin{remark}
    The generating set of groups is synonymous to a generating set for Vector Spaces
\end{remark}
\begin{definition}[Projection]
    Let $G=\langle S,R\rangle$, then a word $w\in S$. projected into $G$ is denoted $[w]$. Moreover we may defined projection equality by
    \begin{align*}
        &[w]=[v] &w\equiv_Gv
    \end{align*}
    
\end{definition}
Although a presentation seems rather abstract, it does have significant value. Because if $G$ has presentation,$\langle S,R\rangle$, then
\begin{itemize}
    \item every word in $G$ can be written as the finite product $x_1\ldots x_n$ with
    \[x_i\in S\cup S^{-1}=\{s^{\pm1}\vert s\in S\}\]
    \item a word $w$ in the alphabet $S\cup S^{-1}$ is equal to the identity element in $G$, $w\equiv_G 1$ if and only if the word $w$ in $F(S)$ is equal a finite product conjugates of the words $r_i\in R$.
    \[ w = \prod_{i=1}^{m}u_i^{-1}r_iu_i\]
    for $m\in\mathbb{N},u_i\in F(S),r_i\in R$. I'm like 99\% sure the book has a typo in it.
\end{itemize}
\begin{lemma}
    If we have $G$ has presentation $\langle S,R\rangle$, and $H$ being a group. We have that $\psi:X\to H$ such that $\psi(r) = 1$ for all $r\in R$, then we may extend $\psi$ to a group homomorphism $\psi: G\to H$ 
\end{lemma}
This was in the book, I have no idea what $X$ is, so I can't really write a proof 
\hypertarget{Exec1_44}{}
\begin{exercise}
    Show that the group $\bigoplus_{x\in X}\mathbb{Z}_2$ has presentation
    \[\langle x\in X\vert x^2,[x,y],\forall x,y\in X\rangle\]
\end{exercise}





\begin{proposition}
    if $G$ has a finite presentation, $\langle S,R\rangle$. If $\langle X,T\rangle$ is an arbitrary presentation of $G$ with $X$ being finite, then there exists a subset of $T$, $T_0\subseteq T$, such that $\langle X,T_0\rangle$ is a presentation of $G$
\end{proposition}
Proof by reference to textbook! But this proposition can be reformulated regarding short exact sequences. Letting $G$ be finitely presented and $X$ to be finite. It follows that $N$ is normally generated by $n_1,\ldots,n_k$ in the following short exact sequence
\[1\to N\to F(X)\to G\to1\]
We may further generalize to arbitrary short exact sequence
\begin{lemma}
    Consider the short exact sequence
    \[1\to N\to K\to G\to1\]
    with $K$ being finitely generated. then $N$ is normally generated by finitely elements $n_1,\ldots,n_k\in N$
\end{lemma}
Proof by reference to textbook!

A list of important groups with finite presentation


\begin{example}[\href{https://en.wikipedia.org/wiki/Fundamental_group}{Fundamental Group}]
    
         The Fundamental Group is presented by $\langle a_1,b_1,\ldots,a_n,b_n\vert [a_1,b_1],\ldots,[a_n,b_n]\rangle $
\end{example}
This group is only \textit{somewhat} useful in \href{https://en.wikipedia.org/wiki/Algebraic_topology}{Algebraic Topology}
\begin{example}[\href{https://arxiv.org/abs/math/0610668}{Right Angled Artin Group}]
    We first let $G=(V,E)$ be a finite graph, then the Right-angled Artin Graph is presented by $\langle V\vert [x_i,x_j] $ whenever $[x_i,x_j]\in E\rangle$
\end{example}
\subsubsection{Coxeter Graph}
\begin{definition}[Coxeter Graph]
    Let $G=(V,E)$ be a \href{https://mathworld.wolfram.com/SimpleGraph.html}{simple} finite graph. then for each edge $e\in E$, we may label it $m(e)\in\mathbb{Z}^+\backslash1$. Then we may call the pair
    \[\Gamma :=(G,m:E\to \mathbb{Z}^+\backslash1)\]
    A \textit{Coxeter Graph}
\end{definition}
\begin{example}[\href{https://en.wikipedia.org/wiki/Coxeter_group}{Coxeter Group}]
    Let $\Gamma$ be a Coxeter Graph, then the corresponding Coxeter Group, $C_\Gamma$, is presented by
    \[\left\langle x_i\in V|x^2_i,(x_ix_j)^{m(e)}, \text{whenever there exists an edge }e=[x_i,x_j]\right\rangle\]
\end{example}
\begin{example}[\href{https://en.wikipedia.org/wiki/Artin-Tits_group}{Artin Groups}]
    I will add all these later!
\end{example}
\begin{example}[\href{https://en.wikipedia.org/wiki/Complex_reflection_group\%23Shephard_groups}{Shephard Group}]
    I will add all these later!
\end{example}
\begin{example}[Generalized von Dyck group]
    I will add all these later!
\end{example}
\begin{example}[Integer Heisenberg group]
    I will add all these later!
\end{example}
\begin{example}[\href{https://en.wikipedia.org/wiki/Baumslag-Solitar_group}{Baumslag–Solitar group}]
I will add all these later!
\end{example}
Up until now a lot of these Groups were defined combinatorically, meaning in terms of their presentations. But below we have several important classes of finitely presented groups which are defined \textit{geometrically}
\begin{itemize}
    \item $CAT(-1)$ Groups
    
    Groups $G$ which act geometrically on $CAT(-1)$ metric spaces.
    \item $CAT(0)$ Groups
    
    Groups $G$ which act geometrically on $CAT(0)$ metric spaces.
    \item Automatic groups
    \item Hyperbolic and relatively hyperbolic groups

    This will be covered in Chapter 11 of GGT, maybe we'll get there?

    \item Semihyperbolic groups
\end{itemize}
\begin{theorem}
    Every finitely generated group is the fundamental group of a smooth closed manifold of dimension 4.
\end{theorem}
This is actually a crazy theorem, like who even came up with this? Are they math god?
\begin{definition}[Laws in Groups]
    An identity or Law is a non-trivial reduced word
    \[w=w(x_1\ldots x_n)\]
    in the letters $x_1,\ldots,x_n$ and their inverses
\end{definition}
I dont understand what this part means, gotta ask vivian!
\subsection{GGT Chapter 7.4 - The Rank of a Free Group Determines The Group}
\begin{proposition}
    Two free groups $F(X)$ and $F(Y)$ are isomorphic if $X$ and $Y$ have the same cardinality
\end{proposition}
\begin{lemma}
    The quotient $\bar{F} := F\backslash N$ is isomorphic to $A=\mathbb{Z}_2^{\bigoplus X}$
\end{lemma}
The proofs for these two are kind of disgusting. But Proposition Proposition 1.58 implies that for every $n\in\mathbb{N}$, up to isomorphism, there exists only one Free Group of rank $n$. We use $F_n$ to denote such a group
\begin{corollary}
    For every $n\in\mathbb{N}$, we have that $n$ is the smallest size for a generating set of $F_n$. Or rather bluntly, rank($F_n$)=$n$
\end{corollary}
\begin{theorem}[Nielsen–Schreier]
    Any subgroup of a free group is a free group.

\end{theorem}
\subsection{GGT Chapter 7.5 - Free Constructions, Amalgams of Groups}
The motivation for this Chapter is that \textbf{Amalgams} (amalgamated free products and HNN extensions) allow for us to build more complex groups when we're either given 2 groups, or a group and a pair of isomorphic subgroups
\begin{definition}[Free Product]
We define the \textit{free product} on groups $G_1=\langle X_1\vert R_1\rangle$ and $G_2=\langle X_2\vert R_2\rangle$. by the representation
\[ G_1\star G_2 = \langle G_1,G_2\vert\hspace{3mm}\rangle\]
Which is short hand for
\[\langle X_1\sqcup X_2\vert R_1\sqcup R_2\rangle\]
\end{definition}
Although this definition is fine, it isn't very flexible as we're restrained to dealing with the entire group.
\begin{definition}[Amalgamated Free Product]
    So we have a more general definition which requires us to take subgroups $H_1\leq G_1$ and $H_2\leq G_2$, along with an isomorphism $\phi: H_1\to H_2$. Then we may define the \textit{amalgamated free product} as
    \[G_1{}_{\star H_1\cong H_2}G_2=\langle G_1,G_2\vert\phi(h)h^{-1},h\in H_1\rangle\]
\end{definition}
the amalgamated free product is very similar to the ``regular" free product. But in addition to the relations defined on $G_1$ and $G_2$, we further define the relation $\phi(h)h^{-1}$ for $h\in H_1$.
\begin{definition}[HHN Extensions]
    This is just a variation of the amalgamated free product, but with the extra hypothesis that $G_1=G_2$. So let $G$ be a group, $H$ be a subgroup, and $\phi:H\to G$ be a monomorphism, then the \textit{HNN Extension} of $G$ via $\phi$ is defined as
    \[G_{\star H,\phi}=\langle G,t\vert tht^{-1}=\phi(h),\forall h\in H\rangle\]
    Where $t$ is just some new generator for the group.
\end{definition}
Again we may further generalize this notion of extension by considering multiple subgroups and multiple isomorphic embeddings
\begin{definition}[Simultaneous HNN Extension]
    Let $J$ be an indexing set, then suppose that we are given a collection of subgroups $H_j$ of $G$ and isomorphic embeddings $\phi_j:H_j\to G$. Then we define the \textit{simultaneous HNN extension} of $G$ to be the group
    \[G\star_{\phi_j:H_j\to G,j\in J}=\langle G,t_j,j\in J\vert t_j h t_j^{-1}=\phi_j(h),\forall h\in H_j,j\in J\rangle\footnote{This might be the most disgusting line of latex that I have ever written}\]
\end{definition}
For this next part we shift our focus onto \textbf{Graphs of Groups}. We no longer assume for our graphs to be simple, but we still assume that they are connected.
\begin{definition}[Graph of Groups]
    Let $\Gamma$ be Graph. assign to each vertex, $v$, in $\Gamma$ a group $G_v$; similarly assign each edge, $e$, to a group $G_e$. Then we may \textit{orient} each edge, $e$, so that it's head is $e_+$ and tail is $e_-$ (Directing each edge). Furthermore suppose that for each edge, $e$, there exists monomorphisms\footnote{You can drop the injectivity requirement, but this won't be explored in this reading}
    \begin{align*}
        &\phi_{e_+} :G_e \to G_{e_+} &\phi_{e_-} :G_e \to G_{e_-}
    \end{align*}
    Then we say that the oriented graph $\Gamma$ together with the collection of vertex and edge groups, along with the monomorphism $G_{e_\pm}$ is called the \textit{graph of groups}, $\mathcal{G}$, based on $\Gamma$.
\end{definition}
